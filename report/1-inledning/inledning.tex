\section{Inledning}
För att anpassa sig till dygnets cykliska beteende använder sig många organismer av en så kallad cirkadisk rytm, vilket är en biologisk klocka med en period på 24 timmar. I en förenklad modell över hur denna klocka fungerar kan rytmen sägas bero på två specifika, reglerande proteiner, ett som undertrycker (repressor, i rapporten nämnd protein R) och ett som aktiverar (activator, protein A) relevanta processer. Undersökningar visar att oscillatorns beteende till största del beror på två faktorer: koncentrationen ac protein R och molekylärdynamiken i processen då protein A bildar ett inaktivt komplex med R \cite{ref:rapport}.

Syftet med detta miniprojekt var att med hjälp av någon av de inbyggda ode-lösarna i \textsc{matlab} lösa masterekvationen given i den vetenskapliga artikel skriven av Vilar \emph{et al.} som låg till grund för uppgiften. En graf över de båda proteinernas beteende skulle tas fram för två olika värden på nedbrytningshastigheten för protein R och valet av ode-lösare skulle motiveras. Slutligen skulle också steglängdens beroende på tiden för de två ode-lösarna \texttt{ode45} och \texttt{ode15s} undersökas.