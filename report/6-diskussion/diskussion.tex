\section{Diskussion}
Som förutspått erhölls ett cykliskt beteende för de studerade proteinernas mängder vid lösande av de givna differentialekvationerna. Den oscillerande rörelsen i figur \ref{fig:GenOsc} uppkommer enligt Vilar \emph{et al.} \cite{ref:rapport} då förloppet som studeras är begränsat till ett slutet system utan stabila punkter. Beroende på vissa parametervärden uppstår inga cykler, utan reaktionen når teoretiskt sett ett stabilt tillstånd redan efter ett ''varv''. Detta förväntas ske då till exempel nedbrytningshastigheten för protein R ($\delta_R$) är tillräckligt hög eller låg och är anledningen till grafens utseende i figur \ref{fig:GenOsc008}. I verkligheten leder dock molekylernas slumpmässiga beteende till att oscillationen fortgår även under dessa förhållanden, något som kan studeras med hjälp av en stokastiskt beräkningsmetod. 

Skriptet \texttt{GenOscSim.m} löser uppgiftens system av ode:er två gånger med hjälp av två olika lösningsmetoder, \texttt{ode45} respektive \texttt{ode15s}. I avsnitt \ref{sec:resultat} ser vi att den första metoden (\texttt{ode45}) tar betydligt längre tid på sig än vad den andra gör. Detta trots att den genomsnittliga tidsåtgången per steg är ca. 15 gånger mindre för \texttt{ode45} (vilket är logiskt då den sistnämnda använder sig av en implicit lösningsmetod, vilket innebär längre beräkningstider per beräkningssteg). Förklaringen till detta visas i figur \ref{fig:Steg48}, där det uppenbaras att \texttt{ode45} använder sig att avsevärt fler steg än vad \texttt{ode15s} gör.