\section{Matematisk modell}
\label{sec:modell}

Systemet kan modelleras som ett system av ordinära differentialekvationer
enligt ekvationssystemet:
\begin{align}
\dd {D_A} t &= \theta_A D'_A - \gamma_A D_A A \\
\dd {D_R} t &= \theta_R D'_R - \gamma_R D_R A \\
\dd {D'_A} t &= \gamma_A D_A A - \theta_A D'_A \\
\dd {D'_R} t &= \gamma_R D_R A - \theta_R D'_A \\
\dd {M_A} t &= \alpha'_A D'_A + \alpha_A D_A - \delta_{M_A}M_A \\
\dd A t &= \beta_A M_A + \theta_AD'_A + \theta_RD'R - A (\gamma_A D_A +
	\gamma_R D_R + \gamma_C R + \delta_A) \\
\dd {M_R} t &= \alpha'_R D'_R + \alpha_R D_R - \delta_{M_R} M_R \\
\dd R t &= \beta_R M_R - \gamma_C A R + \delta_A C - \delta_R R \\
\dd C t &= \gamma_C A R - \delta_A C
\label{eqn:ode}
\end{align}

Detta ekvationssystem är en förenklad modell över hur den cirkadiska rytmen fungerar. Modellen är inte till för att ge specifika detaljer kring hur den biologiska klockan fungerar för samtliga organismer utan för att ge en överblick över processen i allmänhet. För initialvärden och parametrar, se avsnitt~\ref{sec:indata}.

