\section{Resultat}
\label{sec:resultat}

\emph{Kod till samtliga script finns att skåda i sin helhet i Appendix \ref{sec:appendix}.}

Nedan följer ett diary-exempel över en körning av skriptet \texttt{GenOscSim.m}.

\begin{verbatim}
genOscSim

ode45 =

20.3898

Tidsåtgång per steg: 5.6029e-05

ode15s =

0.8554

Tidsåtgång per steg: 0.0007567

diary off
\end{verbatim}

\begin{figure}[H]
\centering
\includegraphics[clip=true,trim=4cm 9cm 5cm 9cm,scale=0.85]{5-resultat/GenOsc}
\caption{Graf över hur de studerade proteinernas mängder varierade över tiden, med parameter- och initialvärden givna i avsnitt \ref{sec:indata} och för $\delta_R = 0,2$ h$^{-1}$.}
\label{fig:GenOsc}
\end{figure}

\begin{figure}[H]
	\centering
	\includegraphics[clip=true,trim=4cm 9cm 5cm 9cm,scale=0.85]{5-resultat/GenOsc008}
	\caption{Graf över hur de studerade proteinernas mängder varierade över tiden, med parameter- och initialvärden givna i avsnitt \ref{sec:indata} och för $\delta_R = 0,08$ h$^{-1}$.}
	\label{fig:GenOsc008}
\end{figure}

\begin{figure}[H]
	\centering
	\includegraphics[clip=true,trim=4cm 9cm 5cm 9cm,scale=0.85]{5-resultat/Steg48}
	\caption{Grafer över hur steglängden varierade över tiden vid lösandet av systemet av ode:er för två olika lösningsmetoder, \texttt{ode45} respektive \texttt{ode15s}.}
	\label{fig:Steg48}
\end{figure}

Variablerna \texttt{ode45} och \texttt{ode15s} är tiden i sekunder som har gått åt för att lösa differentialekvationerna med respektive lösning. Nedanför varje har medelvärdet av tidsåtgången för varje steg skrivits ut.

Figur \ref{fig:GenOsc} visar hur mängden av de båda proteinerna varierar under 200 timmar då hastigheten för den spontana nedbrytningen av protein R ($\delta_R$) är 0.2 h$^{-1}$. Figur \ref{fig:GenOsc008} visar detsamma fast med $\delta_R =$ 0.08 h$^{-1}$.

Figur \ref{fig:Steg48} visar hur tidssteglängden varierar med avseende på tid (i timmar) för de två olika metoderna ode45 respektive ode15s.